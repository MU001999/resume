% !TEX TS-program = xelatex
% !TEX encoding = UTF-8 Unicode
% !Mode:: "TeX:UTF-8"

\documentclass{resume}
\usepackage{zh_CN-Adobefonts_external} % Simplified Chinese Support using external fonts (./fonts/zh_CN-Adobe/)
%\usepackage{zh_CN-Adobefonts_internal} % Simplified Chinese Support using system fonts
\usepackage{linespacing_fix} % disable extra space before next section
\usepackage{cite}

\begin{document}
\pagenumbering{gobble} % suppress displaying page number

\name{许博}

\basicInfo{
  \email{Mu001999@outlook.com} \textperiodcentered\
  \phone{(+86) 185-117-10247} \textperiodcentered\
  \github[Mu001999]{https://github.com/MU001999}}

\section{\faGraduationCap\  教育背景}
\datedsubsection{\textbf{北京林业大学}, 北京}{2016 -- 至今}
\textit{在读本科生}\ 网络工程, 预计 2020 年 6 月毕业

\section{\faUsers\ 实习经历}
\datedsubsection{\textbf{北京市威富安防科技有限公司}, 北京, 中国}{2017年7月 -- 2017年9月}
\role{实习}{系统开发}

\section{\faGithubAlt\ 个人项目}
\datedsubsection{\textbf{Ice Language}}{\url{https://github.com/ice-lang/ice}}
\role{编程语言, 自举, REPL, JIT}{}
\begin{onehalfspacing}
\textbf{项目介绍: } 解释型面向对象程序语言,符合图灵完备思想, 已自举, 提供 REPL ( Read-eval-print-loop, 交互式解释器), 提供丰富的内建类型, 如 map, list 等, 支持运行时 JIT 加速
\end{onehalfspacing}

\datedsubsection{\textbf{CommonRegEx}}{\url{https://github.com/MU001999/commonregex}}
\role{标准语法, DFA, UTF-8 特化}{}
\begin{onehalfspacing}
\textbf{项目介绍: } 为学习正则表达式, 使用 C++ 编写的一个正则表达式引擎, 支持标准语法以及一些扩展语法, 提供通用版本和 UTF-8 特化版本
\end{onehalfspacing}

\datedsubsection{\textbf{Yac, Yet another coroutine}}{\url{https://github.com/MU001999/yac}}
\role{ucontext, 协程}{}
\begin{onehalfspacing}
\textbf{项目介绍: } C++ 实现的一个小型协程库, 使用简单, 提供 create, destroy, resume, yield 等接口
\end{onehalfspacing}

\section{\faObjectGroup\ 合作项目}
\datedsubsection{\textbf{Icarus}}{\url{https://github.com/Jusot/icarus}}
\role{Socket, 并发, I/O 复用, Reactor 模型}{}
\begin{onehalfspacing}
\textbf{项目介绍: } 服务网络编程课设, 合作编写的一个基于 Reactor 模型的并发网络库, 封装了 TcpConnection, 提供 TcpServer 以及线程池等
\end{onehalfspacing}

\datedsubsection{\textbf{Spider Man}}{\url{https://github.com/MU001999/spiderman}}
\role{内容抓取, 多线程, HTTP 解析, HTML 去噪}{}
\begin{onehalfspacing}
\textbf{项目介绍: } Unix/Linux 环境下基于 C++ 编写的互联网内容抓取项目, 实现多线程爬取网页, HTTP 解析以及 HTML 内容去噪分析等
\end{onehalfspacing}

% Reference Test
%\datedsubsection{\textbf{Paper Title\cite{zaharia2012resilient}}}{May. 2015}
%An xxx optimized for xxx\cite{verma2015large}
%\begin{itemize}
%  \item main contribution
%\end{itemize}

\section{\faCogs\ 技能}
% increase linespacing [parsep=0.5ex]
\begin{itemize}[parsep=0.5ex]
  \item \textbf{编程语言}: 较为熟悉 C++, 了解并能使用其较新标准, 亦能较快熟悉其它语言进行开发
  \item \textbf{开发工具}: 掌握 Git 的使用, 熟悉版本控制以及版本管理
  \item \textbf{编译原理}: 较为熟悉前端知识,后端优化涉及部分基础知识
  \item \textbf{数据结构与算法}: 掌握基本数据结构与算法知识, 熟悉常见网络安全知识, 实现过 SAT 协议
\end{itemize}

\section{\faInfo\ 其他}
% increase linespacing [parsep=0.5ex]
\begin{itemize}[parsep=0.5ex]
  \item GitHub: https://github.com/Mu001999
  \item 博客 (基于 Flask + MongoDB 搭建): http://mu00.jusot.com
\end{itemize}

%% Reference
%\newpage
%\bibliographystyle{IEEETran}
%\bibliography{mycite}
\end{document}
