% !TEX TS-program = xelatex
% !TEX encoding = UTF-8 Unicode
% !Mode:: "TeX:UTF-8"

\documentclass{resume}
\usepackage{zh_CN-Adobefonts_external} % Simplified Chinese Support using external fonts (./fonts/zh_CN-Adobe/)
%\usepackage{zh_CN-Adobefonts_internal} % Simplified Chinese Support using system fonts
\usepackage{linespacing_fix} % disable extra space before next section
\usepackage{cite}

\begin{document}
\pagenumbering{gobble} % suppress displaying page number

\name{许博}

\basicInfo{
  \email{Mu001999@outlook.com} \textperiodcentered\ 
  \phone{(+86) 185-1171-0247} \textperiodcentered\ 
  \github[Mu001999]{https://github.com/MU001999}}
 
\section{\faGraduationCap\  教育背景}
\datedsubsection{\textbf{北京林业大学}, 北京}{2016 -- 至今}
\textit{在读本科生}\ 网络工程, 预计 2020 年 6 月毕业

\section{\faUsers\ 工作经历}
\datedsubsection{\textbf{北京市威富安防科技有限公司}, 北京, 中国}{2017年7月 -- 2017年9月}
\role{实习}{深度学习实习生}

\section{\faGithubAlt\ 个人项目}
\datedsubsection{\textbf{Ice Language}}{\url{https://github.com/ice-lang/ice}}
解释型程序语言,符合图灵完备思想
\begin{onehalfspacing}
\begin{itemize}
  \item 面向对象语言
  \item 提供REPL(Read-eval-print-loop, 交互式解释器)
  \item 基本实现自举,ColdIce(由Ice自举实现, https://github.com/ice-lang/coldice)
\end{itemize}
\end{onehalfspacing}

\datedsubsection{\textbf{CommonRegEx}}{\url{https://github.com/MU001999/commonregex}}
使用DFA构造的标准正则表达式库
\begin{onehalfspacing}
\begin{itemize}
  \item 支持标准正则表达式语法
  \item 支持部分扩展语法,如\^, \&等
  \item 提供match, search以及replace等常用API
  \item 另有UTF-8特化版本Yare(https://github.com/MU001999/yare)
\end{itemize}
\end{onehalfspacing}

\datedsubsection{\textbf{Kaleidoscope}}{\url{https://github.com/MU001999/kaleidoscope}}
LLVM Tutorial中Kaleidoscope语言的一个实现
\begin{onehalfspacing}
\begin{itemize}
  \item 支持自定义运算符
  \item 解释执行时支持JIT
  \item 支持编译到object文件
\end{itemize}
\end{onehalfspacing}

% Reference Test
%\datedsubsection{\textbf{Paper Title\cite{zaharia2012resilient}}}{May. 2015}
%An xxx optimized for xxx\cite{verma2015large}
%\begin{itemize}
%  \item main contribution
%\end{itemize}

\section{\faCogs\ 技能}
% increase linespacing [parsep=0.5ex]
\begin{itemize}[parsep=0.5ex]
  \item \textbf{编程语言}: 较为熟悉C++, 亦能较快熟悉其它语言进行开发
  \item \textbf{数据结构与算法}: 掌握基本数据结构与算法知识
  \item \textbf{编译原理}: 较为熟悉前端知识,后端优化涉及部分基础知识
  \item \textbf{编辑器与IDE}: 可以适应任何编辑器, 平时使用VSCode
\end{itemize}

\section{\faHeartO\ 获奖情况}
\datedline{\textit{华北赛区一等奖}, 全国大学生物联网设计竞赛(TI杯)}{2017年8月16日}

\section{\faInfo\ 其他}
% increase linespacing [parsep=0.5ex]
\begin{itemize}[parsep=0.5ex]
  \item GitHub: https://github.com/mu001999
  \item 博客: http://mu00.jusot.com
\end{itemize}

%% Reference
%\newpage
%\bibliographystyle{IEEETran}
%\bibliography{mycite}
\end{document}
