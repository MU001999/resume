% !TEX TS-program = xelatex
% !TEX encoding = UTF-8 Unicode
% !Mode:: "TeX:UTF-8"

\documentclass{resume}
\usepackage{zh_CN-Adobefonts_external} % Simplified Chinese Support using external fonts (./fonts/zh_CN-Adobe/)
%\usepackage{zh_CN-Adobefonts_internal} % Simplified Chinese Support using system fonts
\usepackage{linespacing_fix} % disable extra space before next section
\usepackage{cite}

\begin{document}
\pagenumbering{gobble} % suppress displaying page number

\name{许博}

\basicInfo{
  \email{Mu001999@outlook.com} \textperiodcentered\
  \phone{(+86) 185-117-10247} \textperiodcentered\
  \github[Mu001999]{https://github.com/MU001999}}

\section{\faGraduationCap\  教育背景}
\datedsubsection{\textbf{北京林业大学}, 北京}{2016 -- 至今}
\textit{在读本科生}\ 网络工程, 预计 2020 年 6 月毕业

\section{\faUsers\ 实习经历}
\datedsubsection{\textbf{北京市威富安防科技有限公司}, 北京, 中国}{2017年7月 -- 2017年9月}
\role{实习}{系统开发}

\section{\faGithubAlt\ 个人项目}
\datedsubsection{\textbf{Ice Programming Language}}{\url{https://github.com/ice-lang/ice-lang}}
\role{编程语言, 虚拟机, REPL, 闭包, call/cc}{}
\begin{onehalfspacing}
\textbf{项目介绍: } 解释型程序语言,图灵完备, 提供 REPL ( Read-eval-print-loop, 交互式解释器), 内建常用类型, 如 map, list 等, 支持闭包, call/cc 等高级特性, 可通过 call/cc 实现协程, 亦可通过闭包模拟面向对象
\end{onehalfspacing}

\datedsubsection{\textbf{Parsec}}{\url{https://github.com/MU001999/parsec}}
\role{语法分析器组合子}{}
\begin{onehalfspacing}
\textbf{项目介绍: } 一个 C++ 的语法分析器组合子库, 相当于 C++ 内的 DSL, 可以与 lambda 表达式组合成实际可用的语法分析器, 应用于毕设当中
\end{onehalfspacing}

\datedsubsection{\textbf{CommonRegEx}}{\url{https://github.com/MU001999/commonregex}}
\role{标准语法, DFA, UTF-8 特化}{}
\begin{onehalfspacing}
\textbf{项目介绍: } 使用 C++ 编写的一个正则表达式引擎, 支持标准语法以及一些扩展语法, 提供 ASCII 码版本和 UTF-8 特化版本
\end{onehalfspacing}

\section{\faObjectGroup\ 合作项目}
\datedsubsection{\textbf{Icarus}}{\url{https://github.com/Jusot/icarus}}
\role{Socket, 并发, I/O 复用, Reactor 模型}{}
\begin{onehalfspacing}
\textbf{项目介绍: } 服务网络编程课设, 合作编写的一个基于 Reactor 模型的并发网络库, 封装了 TcpConnection, 提供 TcpServer 等
\end{onehalfspacing}

\datedsubsection{\textbf{Maaath}}{\url{https://github.com/MU001999/maaath}}
\role{公式解析, 倒排索引, 搜索引擎, 中文分词}{}
\begin{onehalfspacing}
\textbf{项目介绍: } Unix/Linux 环境下使用 C++ 编写的针对教学场景的搜索引擎, 实现中文分词, 公式解析, 以及基于倒排索引的搜索引擎
\end{onehalfspacing}

% Reference Test
%\datedsubsection{\textbf{Paper Title\cite{zaharia2012resilient}}}{May. 2015}
%An xxx optimized for xxx\cite{verma2015large}
%\begin{itemize}
%  \item main contribution
%\end{itemize}

\section{\faCogs\ 技能}
% increase linespacing [parsep=0.5ex]
\begin{itemize}[parsep=0.5ex]
  \item \textbf{编程语言}: 较为熟悉 C++, 了解并能使用其较新标准, 亦能较快熟悉其它语言进行开发
  \item \textbf{开发工具}: 掌握 Git 的使用, 熟悉版本控制以及版本管理
  \item \textbf{网络编程}: 熟悉基础的网络编程知识以及多线程等操作
\end{itemize}

\section{\faInfo\ 其他}
% increase linespacing [parsep=0.5ex]
\begin{itemize}[parsep=0.5ex]
  \item 知乎: https://zhuanlan.zhihu.com/3thinks
  \item 博客: http://mu00.jusot.com
\end{itemize}

%% Reference
%\newpage
%\bibliographystyle{IEEETran}
%\bibliography{mycite}
\end{document}
